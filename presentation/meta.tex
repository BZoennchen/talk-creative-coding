%Meta-Informationen%

%\newcommand{\mylogo}{./hm_logo_alt}
\newcommand{\mydate}{\today}

%\addto\captionsngerman{%
%\renewcommand{\figurename}{Abb.}%
%\renewcommand{\tablename}{Tab.}%
%}
\graphicspath{{pictures/}}
\hypersetup{pdfauthor={Benedikt Sebastian Zönnchen}}
\hypersetup{pdftitle={Efficient parallel algorithms for large-scale pedestrian simulation}}

\newcommand{\shotAbstractEng}{
The understanding and prevention of catastrophes at large-scale events are of utmost societal importance. 
For that, pedestrian simulation has proven to be a potent tool.
Using microscopic pedestrian simulations, researchers and practitioners investigate the mechanisms and preconditions that lead to dangerous situations such as harmful crowd pressures.
Simulations reveal the behaviors and characteristics of human crowds and suggest practical ways to prevent catastrophes.

However, microscopic pedestrian simulations are computationally expensive.
Yet, it is necessary to model each individual to predict crucial phenomena.
Despite their computational cost, real-time simulations are required to make reliable predictions during ongoing events and to enhance a research field that integrates more and more data-driven methods.
To achieve this temporal requirement, we have to introduce and exploit efficient and parallel algorithms.

In this thesis, I follow the call for efficient and scalable simulations by analyzing existing and introducing new parallel algorithms.
I introduce parallelism to a class of microscopic models, i.e., optimal steps models, and show that real-time simulations of half a million participants are possible.
In addition, I develop efficient and parallel algorithms to construct navigation fields: a robust technique to model pedestrian wayfinding.
A new meshing algorithm reduces the problem size and a novel numerical method exploits similarities of consecutively solved eikonal equations.
In combination, real-time dynamic navigation field computation becomes possible for many large-scale scenarios.
}

\newcommand{\shotAbstractGer}{
Katastrophen inmitten von Großveranstaltungen verstehen und verhindern ist von größter ge\-sell\-schaft\-lich\-er Bedeutung.
Für diese Aufgabe haben sich Fußgängersimulationen als wirksames Werkzeug bewährt.
Mit ihrer Hilfe unter\-such\-en Forscher und unmittelbare Anwender die Voraussetzungen und Zusammenhänge, die zu gefährlichen Situationen, wie kritischen Stauungen, führen.
Aus den gewonnenen Erkenntnissen über das Verhalten und der Bewegung von Fußgängern, können wir praktische Maßnahmen ableiten und dadurch Katastrophen verhindern.
%Durch die Erkenntnisse, die wir über das Verhalten großer Menschenmengendie aus Simulationen gewinnen, können wir praktische Maßnehmen entwickelt und Katastrophen verhindern.

Mikroskopische Fußgängersimulationen sind jedoch rechenintensiv.
Um aussagekräftige Vor\-her\-sagen zu erzielen, ist, bis heute, die Modellierung jedes einzelnen Individuums erforderlich.
%um entscheidende Phä\-no\-me\-ne vorherzusagen
Trotz des dadurch entstehenden Rechenaufwands, wird der Ruf nach Echtzeitsimulationen immer lauter.
%Zugleich sind, trotz des entstehenden Rechenaufwands, Echtzeitsimulationen erstrebenswert.
Einerseits sollen sie die Anwender mit Vorhersagen während einer laufenden Veranstaltung unterstützen.
%Einerseits sollen sie, bei gerade lauf\-en\-den Er\-eig\-nis\-sen, zuverlässige Vorhersagen ermöglichen.
Andererseits wür\-den sie ein Forschungsfeld bereichern, welches immer mehr daten\-ge\-trie\-bene Methoden integriert.
Um diese zeitliche Anforderung zu erfüllen, müssen wir effiziente und parallele Algorithmen für die Be\-rech\-nung von Per\-so\-nen\-strömen entwickeln und nutzen.

In dieser Arbeit folge ich dem Ruf nach effizienten und skalierbaren Simulationen, indem ich vorhandene Algorithmen analysiere und neue parallele Algorithmen entwickle.
Zunächst führe ich die Parallelität in die sogenannten Optimal Steps Modelle, eine Klasse mikroskopischer Modelle, ein.
Ich zeige, dass dadurch die Simulation einer halben Million (virtueller) Fuß\-gäng\-er in Echtzeit möglich wird.
Darüber hinaus entwickle ich effiziente und parallele Algorithmen zur Berechnung von Navigationsfeldern.
Diese haben sich in der Vergangenheit als robuste Technik zur Modellierung der Weg\-fin\-dung von Fußgängern etabliert.
Ein neuer Algorithmus zur Netzgenerierung reduziert die Größe des zu berechnenden Problems und eine neuartige numerische Methode nutzt die Ähnlichkeit aufeinanderfolgend gelöster Eikonalgleichungen aus.
In Kombination wird der Einsatz dynamischer Navigationsfelder für Echtzeitsimulationen für viele große Sze\-na\-rien ermöglicht.
}

\newcommand{\shotAbstractEngB}{
In an ever more connected world, large-scale events with millions of participants are culturally significant and offer joy, happiness, and a place where the human spirit can grow. 
It is sholcing when things go wrong, and joy turns into harm – when our belonging for connection destroys them forever.
We must learn from accidents of the past, find their causes, and develop mechanisms to prevent them in the future.
Therefore, we study human behavior supported by microscopic pedestrian simulation.
Today, real-time simulations are required to predict the future of an ongoing event and to enhance a research field that integrates more and more data-driven methods.
However, simulations are computationally expensive, and the only way to achieve scalability is parallelization.

In this thesis, I follow the call for efficient and scalable simulations by analyzing existing and introducing new parallel algorithms.
I introduce parallelism to a class of microscopic models, \ie{}, \OSMs{} and show that real-time simulations of half a million participants is possible.
In addition, I develop efficient and parallel algorithms to construct navigation fields -- a robust technique to model the wayfinding of pedestrians.
A new meshing algorithms reduces the problem size and a noval numerical method exploits similarities of consecutive solved eikonal equations.
In combination real-time dynamic navigation field computation becomes possible for many large-scale scenarios.
}

\newcommand{\shotAbstractGerB}{
In einer immer kleiner werdenden Welt, sind Großveranstaltungen mit Millionen von Teilnehmern ein wichtiges kulturelles Gut.
Sie sind eine Quelle des Glück, der Freude und bieten einen Platz an dem der menschliche Geist wachsen kann.
Umso schrecklicher ist es anzusehen, wenn diese Freude in Schaden mündet und unsere Verlangen noch Verbundenheit diese für immer zerstört.
Wir müssen aus vergangenen Fehlern lernen, ihren Ursprung ergründen und zukünftige Unfälle durch präventive Maßnahmen verhindern.
Deshalb studieren wir mittels mikroskopischer Personenstromsimulationen menschliches Verhalten.
Heute benötigen wir Echtzeitsimulationen um die Entwicklung einer anhaltenden Großveranstaltung vorherzusagen.
Zudem helfen uns effiziente Simulationen die aufkommenden datengetriebenen Methoden in unseren Forschungsbereich zu integrieren.
Doch sind mikroskopische Simulationen rechenintensiv und um Skalierbarkeit zu erzielen braucht es Parallelität.

In dieser Arbeit folge ich dem Ruf nach effizienten und skalierbaren Simulationen.
Hierzu analysiere ich existierende Algorithmen und stelle neue parallele Algorithmen vor.
Ich parallelisiere eine ganze Klasse von Personenstrommodellen, nämlich die sogenannten Optimal Steps Modelle.
Ich zeige, dass die Simulation von einer halben Millionen Teilnehmern in Echtzeit möglich ist.
Zusätzlich entwickle ich effiziente und parallele Algorithmen zur Konstruktion von Navigationsfeldern.
Sie sind eine robuste Technik um die Wegfindung von Fußgängern zu modellieren. 
Ein neuer Algorithmus zur Gittererzeugung reduziert die zu lösende Problemgröße und eine neue numerische Methode nutzt Ähnlichkeiten in den aufeinanderfolgend zu lösenden Eikonalgleichungen.
In Kombination werden Echtzeitberechnungen von dynamischen Navigationsfeldern für viele großskalige Simulationsszenarien ermöglicht.
}

\newcommand{\longAbstractEng}{
The understanding and prevention of catastrophes at large-scale events are of utmost societal importance. 
For that, pedestrian simulation has proven to be a potent tool.
Using microscopic pedestrian simulations, researchers and practitioners investigate the mechanisms and preconditions that lead to dangerous situations such as harmful crowd pressures.
Simulations reveal the behaviors and characteristics of human crowds and suggest practical ways to prevent catastrophes.

However, microscopic pedestrian simulations are computationally expensive.
Yet, it is necessary to model each individual to predict crucial phenomena.
Despite their computational cost, real-time simulations are required to make reliable predictions during ongoing events and to enhance a research field that integrates more and more data-driven methods.
To achieve this temporal requirement, we have to introduce and exploit parallelism efficiently.

In this thesis, I follow the call for efficient and scalable simulations by analyzing existing and introducing new parallel algorithms.
At first, I give an overview of microscopic models and navigation fields -- a common technique many models employ.
I identify \OSMs{} to be especially important since their foundation is motivated by social psychology and biomechanics.
I then discuss the relation between parallelism and the behavior of individuals within a large crowd.
Instead of enforcing parallelism by compromising the models, I exploit its natural occurrence.
My \textsc{ParallelEventDrivenUpdate} introduces parallelism into \OSMs{} without changing their definition.
I analyze and use the parallelism defined by the models and show that we can simulate up to half a million agents in real-time on affordable off-the-shelf graphics processing units (GPUs).
In the last part of my thesis, I focus on efficient navigation field computation.
In other words, I focus on efficient methods to solve (multiple) eikonal equations.
Navigation fields realize robust wayfinding to facilitate simulations with complex and large geometries.
First, I enter the area of computational geometry and consider different space discretizations.
I then develop \eikmesh{}, a new meshing algorithm for high-quality unstructured two-dimensional unstructured triangular meshes.
The size of the underlying mesh directly influences the time complexity of numerical solvers.
Therefore, \eikmesh{} constructs a mesh with a localized resolution.
Besides reducing the problem size, I develop the \textsc{InformedFastIterativeMethod}, a novel numerical method that exploits consecutive solved eikonal equations.
It is suitable to efficiently compute dynamic navigation fields and might be the source of new numerical methods to solve the eikonal equation in general.
%
%I introduce parallelism to a class of microscopic models, \ie{}, \OSMs{} and show that real-time simulations of half a million participants is possible.
%In addition, I develop efficient and parallel algorithms to construct navigation fields -- a robust technique to model the wayfinding of pedestrians.
%A new meshing algorithms reduces the problem size and a noval numerical method exploits similarities of consecutive solved eikonal equations.
%In combination real-time dynamic navigation field computation becomes possible for many large-scale scenarios.
}

\newcommand{\longAbstractEngB}{
Human beings are social creatures longing for deep emotional connections.
%We find expressions of this basic human need across different cultures.
Today, large-scale events with millions of participantsa are a manifestation of this desire.
In an ever more connected world, they are culturally significant and offer joy, happiness, and a place where the human spirit can grow. 
It is crushing when things go wrong, and joy turns into harm – when our belonging for connection destroys them forever.
We must learn from accidents of the past, find their cause, and develop mechanisms to prevent them in the future.
Therefore, we study the behavior of humans, their decision-making, and motion.
Computer simulations are an essential part of this effort.
They enable researchers and practitioners to analyze, plan, and manage large-scale events.

Microscopic pedestrian simulations are computationally expensive, but it is necessary to model each individual to observe crucial phenomena. 
Today, real-time simulations are required to predict the future of an ongoing event and to enhance a research field that integrates more and more data-driven methods.
In this thesis, I follow this call by analyzing existing and introducing new efficient parallel algorithms to enable large-scale microscopic pedestrian simulations.

At first, I give an overview of microscopic models and a critical technique they are based on, that is, navigation fields.
I identify \OSMs{} to be especially important since their foundation is motivated by social psychology and biomechanics.
I then discuss the relation between parallelism and the behavior of individuals within a large crowd.
Instead of enforcing parallelism by compromising the models, I exploit its natural occurrence.
My \textsc{ParallelEventDrivenUpdate} introduces parallelism into \OSMs{} without changing their definition.
I analyze and use the parallelism defined by the models and show that we can simulate up to half a million agents in real-time on affordable off-the-shelf graphics processing units (GPUs).
In the last part of my thesis, I focus on efficient navigation field computation.
In other words, I focus on efficient methods to solve (multiple) eikonal equations.
Navigation fields realize robust wayfinding to facilitate simulations with complex and large geometries.
First, I enter the area of computational geometry and consider different space discretization.
I then develop \eikmesh{}, a new meshing algorithm for high-quality unstructured $2$-d unstructured triangular meshes.
The size of the underlying mesh directly influences the time complexity of numerical solvers.
Therefore, \eikmesh{} constructs a mesh with a localized resolution.
Besides reducing the problem size, I develop \textsc{InformedFastIterativeMethod}, a novel numerical method that exploits consecutive solved eikonal equations.
It is suitable to efficiently compute dynamic navigation fields and might be the source of new numerical methods to solve the eikonal equation in general.
}

\newcommand{\longAbstractGer}{
Als soziale Wesen streben wir Menschen nach tief emotionalen Verbindungen.
%Ausdrücke dieses Grundbedürfnisses finden sich in verschiedenen Kulturen.
Heute manifestiert sich dieses Verlangen durch Großveranstaltungen mit Millionen von Teilnehmern.
In einer immer kleiner werdenden Welt, sind Sie ein kulturelles Gut und eine Quelle des Glück, der Freude und ein Platz an dem der menschliche Geist wachsen kann.
Umso schrecklicher ist es, wenn diese Freude in Schaden mündet und unsere Verlangen noch Verbundenheit diese für immer zerstört.
Wir müssen aus vergangenen Fehlern lernen, ihren Ursprung ergründen und zukünftige Unfälle durch präventive Maßnahmen verhindern.
Deshalb studieren wir mittels mikroskopischer Personenstromsimulationen menschliches Verhalten.
Sie ermöglichen es Forschern und Anwendern aus der Praxis Großveranstaltungen zu analysieren, zu planen und besser zu lenken. 

Mikroskopische Personenstromsimulationen sind rechenintensiv, doch ist die Mo\-del\-lier\-ung von einzelnen Individuen notwendig um kritische Phenomene beobachten zu können.
Heute benötigen wir Echtzeitsimulationen, um die Zukunft ablaufender Großveranstaltungen vorherzusagen und die aufkommenden datengetriebene Methoden für uns nutzbar zu machen.
In dieser Arbeit folge ich diesem Ruf.
Ich analysiere existierende und entwickle neuer effizienter paralleler Algorithmen die großskalige Echzeitsimulationen ermöglichen.

Zunächst gebe ich einen Überblick über mikroskopische Personenstrommodelle sowie die Modellierung der Wegfindung durch Navigationsfelder.
Da Optimal Steps Modelle sich auf Erkenntnissen aus der Psychologie und der Biomechanik gründen, identifiziere ich sie als besonders wichtig.
Ich diskutiere den Zusammenhang zwischen Parallelität und dem Verhalten der Individuen innerhalb großer Menschenmengen.
Anstatt Parallelität durch Kompromittierung der Modelle zu erzwingen, nutze dessen natürliches Vorkommen.
\textsc{ParallelEventDrivenUpdate} parallelisiert Optimal Steps Modelle ohne dabei Änderungen an den Modellen vorzunehmen.
Ich analysiere und nutze die Parallelität des bestehenden Modells und zeige dass Echzeitsimulationen mit bis zu einer halben Millionen Agenten auf gängigen Grafikkarten möglich sind.
Im letzten Teil meiner Arbeit, konzentriere ich mich auf die effiziente Berechnung der Navigationsfelder.
In anderen Worten, ich konzentriere mich auf effiziente Methoden um (mehrere) Eikonalgleichungen zu lösen.
Navigationsfelder realisieren eine robuste Wegfindung um Simulationen in einer geometrisch komplexen Umgebung zu ermöglichen.
Zu allererst betrete ich den Bereich der algorithmischen Geometrie und betrachte unterschiedliche Raumdiskretisierungen.
Danach entwickle ich \eikmesh{}, einen neuen Gittergenerierungsalgorithmus für qualititiv hochwertige unstrukturierte zweidimensionale Dreieckgitter.
Da die Gittergröße sich direkt auf die Laufzeit numerischer Verfahren auswirkt, konstruiert \eikmesh{} ein Gitter mit lokalisierter Gitterauflösung.
Neben der Reduzierung der Problemgröße nutzt \textsc{InformedFastIterativeMethod} die Ähnlichkeit der aufeinanderfolgend zu lösenden Eikonalgleichungen ausnutzt. Dieses neue Verfahren eignet sich zur effizienten Berechnung dynamischer Navigationsfelder und ist möglicherweise die Quelle weiterer numerischer Methoden zur Lösung der Eikonalgleichung.
}
%\tikzstyle{every node}=[circle, draw, fill=black!50, inner sep=0pt, minimum width=4pt].palenta@tum.de